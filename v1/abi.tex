%! Author = Alexander Hinze
%! Date = 16.11.2023

Due to the need of mangling, the Ferrous compiler defines a simple mangling model
which is similar to the one of C++; each type has a mangled name and every function
name will be transformed based on the parameter types.
\noindent
Each builtin type in the language maps to a hard-coded short-hand name:

\begin{center}
    \begin{tabular}{ |c|c| }
        \hline
        \textbf{Builtin} & \textbf{Mangled Name}\\
        \hlineB{2.5}
        i8 & \color{type_class_sint} sB \normalcolor\\
        i16 & \color{type_class_sint} sS \normalcolor\\
        i32 & \color{type_class_sint} sI \normalcolor\\
        i64 & \color{type_class_sint} sL \normalcolor\\
        \hline
        u8 & \color{type_class_uint} uB \normalcolor\\
        u16 & \color{type_class_uint} uS \normalcolor\\
        u32 & \color{type_class_uint} uI \normalcolor\\
        u64 & \color{type_class_uint} uL \normalcolor\\
        \hline
        f32 & \color{type_class_ieee} F \normalcolor\\
        f64 & \color{type_class_ieee} D \normalcolor\\
        \hline
        isize & \color{type_class_size} sZ \normalcolor\\
        usize & \color{type_class_size} uZ \normalcolor\\
        \hline
        void & \color{type_class_misc} V \normalcolor\\
        bool & \color{type_class_misc} T \normalcolor\\
        char & \color{type_class_misc} C \normalcolor\\
        \hline
    \end{tabular}
\end{center}

\noindent
If a derived type is used, \textbf{P} is appended to the mangled name of the base type
if the derived type attribute indicates a pointer, otherwise \textbf{R} shall be appended to
indicate a reference.
\noindent
This would mean that a function with the name \textbf{foo}
and the parameter types \textbf{usize}, \textbf{usize}, \textbf{\&i32} and \textbf{**void}
would have a mangled name of \textbf{foouZuZsIRVPP}.